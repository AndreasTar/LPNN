Modern interactive 3D applications, like video games, VR/AR apps, simulators etc., depend on believable lighting interactions with the objects of a 3D scene to achieve the desired visual goals, while trying to maintain real-time frame-rate budgets, typically above 30 Frames per Second (FPS). Achieving visual fidelity and performance can be a difficult task and sometimes impossible with the given hardware specifications of the device. For that reason, modern real-time rendering engines, e.g. Unity, Unreal Engine, Godot and others, depend on a number of methods to balance those metrics. 

The illumination of any scene can be split into two very simple categories. Direct Illumination, the light that travels unoccluded from a light source to a surface of an object, is typically handled with techniques like shadow-mapping or screen-space shadows, yielding crisp, high-framerate-capable shadows, but lack in inter-surface light transport situations. In contrast, Indirect Illumination, or Global Illumination (GI), captures light that has bounced or refracted off one or more surfaces, producing soft shadows, color bleeding, and contextually rich shading. 

The field-standard for accurate lighting and shadows in a scene is Path-Tracing, a method that tracks every light ray and any interactions it has with the objects of a 3D scene and calculates the resulting color for each pixel of the screen. Such approach remains prohibitively expensive for most interactive applications, so real-time systems employ precomputation and approximation of the illumination of the scene; static geometry is baked into lightmaps that store per-texel irradiance, while dynamic elements sample from irradiance volumes or light probes, sparse 3D points whose spherical-harmonic coefficients are interpolated at runtime. 

Screen-space GI methods typically approximate a limited number of light-ray bounces directly from the camera's depth buffer, but suffer from missing contextual information outside the camera's view frustum and temporal instability. Voxel-based approaches (e.g., cone-tracing through a low resolution 3D grid) % citation?
enable more dynamic multi-bounce effects at the cost of memory, processing cost and potential blurring of fine detail. 

Across all these techniques, the central challenge is allocating a strict millisecond-scale budget to indirect illumination while maintaining consistency across static and dynamic scene content, avoiding visible seams when blending baked and runtime solutions and fitting within GPU memory constraints. 

Light probes, in particular, represent a compelling middle-ground, flexible enough to illuminate moving objects without rebaking yet compact enough for real-time evaluation, making their optimal placement a critical factor in any high-quality GI pipeline.

% citations where needed?
% fix typos if any, better wording at 1 and 2 paragraph?

\pagebreak % temporary potentially	

\section{Related Works}
There is an abundance of work in the literature addressing the problem of Global Illumination. These studies aim to achieve realistic lighting in 3D scenes by employing various approaches and techniques, each offering unique advantages and disadvantages, but they share a common goal: to minimize computational cost while maximizing visual fidelity.

\subsection{Offline Methods}
Offline Illumination methods refer to techniques that are not viable for real-time applications and are therefore used only in situations where the importance of high visual fidelity far outweighs the need for computational speed, typically in non-interactive 3D renders, most commonly in movies or pre-rendered scenes. Classic Path-Tracing, first introduced in 1986 \parencite{Kajiya1986}, tracks the movement of a photon ray emitted from a source, typically the camera, and simulates physics interactions to calculate the color of each screen pixel accurately. The immense computational cost of path-tracing led to the development of performance improvements, such as the Metropolis Light Transport (MLT) method introduced in 1997 \parencite{Veach1997}, and variants like bi-directional Path-Trace \parencite{Lafortune1993}, which build on Monte-Carlo algorithms \parencite{Lafortune1996}.

\subsection{Online Methods} % maybe section these?

\subsection*{non-AI based methods}

\subsection*{AI based methods}


\section{Thesis Structure}
TODO