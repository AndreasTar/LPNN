\gr Ο αληθοφανής φωτισμός είναι ο ακρογωνιαίος λίθος των οπτικά ελκυστικών τρισδιάστατων γραφικών. Το σύστημα φωτο-ανιχνευτών (\en Light-Probe) \gr της μηχανής γραφικών \en Unity \gr παρέχει έναν αποδοτικό τρόπο καταγράφης και παρεμβολής προπαρασκευασμένων δεδομένων παγκόσμιου φωτισμού (\en Global Illumination) \gr προς όλα τα δυναμικά αντικείμενα μιας σκηνής. Πάραυτα, η χειροκίνητη τοποθέτηση των  φωτο-ανιχνευτών σε πολύπλοκες σκηνές είναι μια χρονοβόρα διαδικασία, αλλά και επιρρεπής σε λάθη, δημιουργώντας μεγάλες καθυστερήσεις κατά την διάρκεια κατασκευής τρισδιάστατων εφαρμογών. Η διπλωματική αυτή παρουσιάζει μία αυτοματοποιημένη μέθοδο βαθιάς μάθησης η οποία προβλέπει τον βαθμό σημαντικότητας ανά σημείο για τα σημεία τοποθέτησης των φωτο-ανιχνευτών, χρησιμοποιώντας ένα νευρωνικό δίκτυο εμπνευσμένο από το \en PointNet.

\gr Αρχικά δημιουργούμε ένα κανονικό τρισδιάστατο πλέγμα σημείων το οποίο προσαρμόζεται στα αυθαίρετα δομημένα όρια της σκηνής, ορισμένα απο τον χρήστη. Δειγματολειπτούμε  πληροφορίες φωτισμού ανά σημείο, συμπεριλαμβάνοντας σφαιρικές αρμονικές, διακυμάνσεις φωτισμού, κανονικών επιφάνειας, και \en RGB, \gr όπως και παράγωντα απόφραξης. Τα χαρακτηριστικά αυτά εμπεριέχουν σημαντικές πληροφορίες που ωθούν την ακρίβεια του παγκόσμιου φωτισμού. Στην συνέχεια τα δεδομένα δειγματοληψίας μετατρέπονται σε ένα συνεκτικό διάνυσμα χαρακτηριστικών ανά σημείο, και χρησιμοποιούνται για την εκπαίδευση του \en PointNet-\gr τύπου μοντέλου τεχνητής νοημοσύνης. Το μοντέλο αυτό καταναλώνει μία αυθαίρετου μήκους λίστα από συνεκτικά διανύσματα χαρακτηριστικών και εξάγει μια πιθανότητα σε εύρος 0 έως 1, η οποία αναπαριστά την κρισιμότητα τοποθέτησης ενός  φωτο-ανιχνευτή σε κάθε σημείο στο πλέγμα.

\gr Στην μηχανή \en Unity, \gr το εκπαιδευμένο μοντέλο εξάγεται σε ένα αρχείο τύπου \en \verb*|.ONNX| \gr και εισάγεται μέσω του \en Sentis, \gr του επίσημου πακέτου της \en Unity \gr για χειρισμό μοντέλων τεχνητής νοημοσύνης εντός του εκτελέσιμου της \en Unity˙ \gr κατα την επεξεργασία, καταναλώνει δεδομένα σκηνής ανά σημείο και επιστρέφει τιμές κρισιμότητας ανά σημείο. Οι προβλεπόμενες τοποθεσίες υψηλής σημασίας χρησιμοποιούνται για να συμπληρώσουν μια ομάδα φωτο-ανιχνευτών, αντικείμενο της \en Unity, \gr παρέχοντας στους χρήστες άμεση και οπτικά κατάλληλη κατανομή των φωτο-ανιχνευτών, με ευκολόχρηστη κατωφλίωση όταν υψηλότερης ή χαμηλότερης σημασίας τοποθεσίες είναι επιθυμητές.

\gr Επιδεικνύουμε ότι το τεχνητής νοημοσύνης μοντέλο μας γενικεύει ανάμεσα σε πλέγματα με διάφορα μεγέθη και σχήματα, χωρίς την ανάγκη επανεκπαίδευσης, όπως και ότι παρέχει άμεσα αποτελέσματα για κάθε σκηνή. Παρόλο που η αξιολόγησή μας παραμένει κυρίως ποιοτική, βασιζόμενη στον οπτικό έλεγχο του αποτελέσματος παγκόσμιου φωτισμού, όπως και τα σημεία τοποθέτησης των φωτο-ανιχνευτών σε ένα εύρος από εσωτερικών και εξωτερικών χώρων, παρατηρούμε συστηματικά πως η παραγόμενη διάταξη των φωτο-ανιχνευτών περιλαμβάνει σημαντικά δεδομένα φωτισμού της σκηνής με ελάχιστη ή μηδενική χειροκίνητη προσαρμογή. Αντικαθιστώντας την χειροκίνητη τοποθέτηση των φωτο-ανιχνευτών με την απλή χρήση ενός μοντέλου τεχνητής νοημοσύνης, οι καλλιτέχνες και οι προγραμματιστές κερδίζουν χρόνο και επιταχύνουν την διαδικασία ανάπτυξη μιας τρισδιάστατης εφαρμογής.\en
