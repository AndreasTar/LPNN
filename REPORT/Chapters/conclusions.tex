In this thesis, a method is implemented to accelerate the placement of light probes in a 3D scene during the development process. The method is experimented on, and the results are presented. The LPNN method and its normal usage are described. Additionally, the process of collecting the required features and labels needed for retraining the Neural Network is also presented. This approach assigns an importance score to a grid-like layout of Evaluation Points, which are then used to place light probes in the scene, depending on a variable threshold value controlled by the developer. Finally, quantitative and qualitative results are presented. 

In the experiments conducted, we concluded that the Neural Network approach is capable of reducing the time needed by the developer for a sufficient light probe layout, as well as minimizing the amount of manual tweaking required for optimal results. The results present a solution that approaches a theoretical optimal light probe layout. While not free of oversampling or undersampling, the LPNN tool provides a consistently suitable layout within a fraction of the time needed by other methods, including the manual approach.

Moving forward from this thesis, an improved Neural Network architecture can be explored, implemented and experimented on. An approach that takes into consideration the edges of each light probe can potentially improve the accuracy of the model presented. While a Deep Learning approach requires a large dataset to be able to generalize sufficiently, an improved architecture and implementation can reduce the cost and improve accuracy by providing more variety in the training data. Additionally, a heuristic approach can be combined, removing or adding light probes in areas that the model inferred to be of low-significance, allowing for results closer to the theoretical optimal layout. Lastly, a new approach can be explored that combines the LPNN approach with the Neural Light Field Probes method \parencite{You2024}, ensuring that the contributions of both methods are preserved.