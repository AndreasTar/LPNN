In this chapter we introduce some basic, required background for this thesis. First, we introduce light probes and the mathematical equations that define them. Then, we present the AI architecture that was the basis of our AI model. Finally, the tools and technologies used for this thesis are presented.

\section{Light Probes}
As mentioned previously, the idea of using discrete probes to capture scene lighting data traces back to early GI research. \cite{Greger1998} introduced the irradiance volume, a 3D grid of sample points storing the irradiance field to approximate GI in complex scenes. A light probe samples the incident radiance at a point in empty space from all directions. Often just the diffuse component of the radiance is captured, since it most commonly varies smoothly, so it can be compactly represented by projecting the lighting onto a truncated spherical harmonic (SH) basis. Third-order SH is most commonly used, storing 9 coefficients per color channel, abbreviated to L2-SH.

\section{Spherical Harmonics}
Spherical Harmonics (SH), first introduced by Pierre Simon de Laplace, are a method of storing information on a point in space. They are categorized in orders, starting from first order. We are interested in third-order SH, since they represent a good middle ground between storage size, computational cost and accuracy. SH are often described as the Fourier Series of functions on the surface of a sphere, breaking down any pattern of light on a sphere into a set of basis frequencies. The order of SH depicts the amount of data we capture, third-order SH, noted as L2 SH, store the first three bands of data, resulting in 9 coefficients per color channel. Bands represent the individual frequencies; the Zeroth band captures the overall average lighting present in that position in space, the First band captures simple directional gradients, and the Second band captures quadratic variations, e.g. gentle light gradients and their shadows.

\section{PointNet}
TODO

\section{Tools}
TODO
